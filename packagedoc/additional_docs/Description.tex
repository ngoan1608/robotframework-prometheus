% --------------------------------------------------------------------------------------------------------------
%
% Copyright 2020-2024 Robert Bosch GmbH

% Licensed under the Apache License, Version 2.0 (the "License");
% you may not use this file except in compliance with the License.
% You may obtain a copy of the License at

% http://www.apache.org/licenses/LICENSE-2.0

% Unless required by applicable law or agreed to in writing, software
% distributed under the License is distributed on an "AS IS" BASIS,
% WITHOUT WARRANTIES OR CONDITIONS OF ANY KIND, either express or implied.
% See the License for the specific language governing permissions and
% limitations under the License.
%
% --------------------------------------------------------------------------------------------------------------

\section{Test setup}

How many tests are passed? How many tests are failed? On which test benches do these tests run? Did any resets occur during test execution?
How about the temporary unavailability of required test external components?

To monitor all these informations, a test system setup is necessary consisting of at least the following components:

\begin{enumerate}
   \item A test framework that executes the test and provides the data to be monitored (here: the \textbf{Robot Framework}).
         This includes the possibility that more than one test framework runs in parallel.

   \item A component that collects and stores data from all executed test frameworks (here: the monitoring system \textbf{Prometheus}).

   \item A component that visualizes all data that have been collected (here: \textbf{Grafana}).
\end{enumerate}

This is not the only way to establish such a test system, but in this document we concentrate on \textbf{Robot Framework}, \textbf{Prometheus} and \textbf{Grafana}.


To be able to collect values, \textbf{Prometheus} requires an http based counterpart. And the \textbf{Robot Framework} must be enabled to support this counterpart.
In pure Python this is realized by a \textbf{Prometheus Python client library}. On \textbf{Robot Framework} level this is done by this component \pkg\ that is a mapping
between the interface of the \textbf{Prometheus Python client library} and \textbf{Robot Framework} keywords.

Or in other words: \pkg\ uses the \textbf{Prometheus Python client library} to provide values to \textbf{Prometheus} to enable the data visualization in \textbf{Grafana}.

The \textbf{Prometheus Python client library} is part of the installation dependencies of \pkg. \textbf{Prometheus} and \textbf{Grafana} are components that
have to be downloaded, installed and configured separately.

% --------------------------------------------------------------------------------------------------------------

\section{Installations}

\textbf{1. Prometheus}

To install \textbf{Prometheus} please visit the \href{https://prometheus.io/}{homepage}. This homepage also contains a \textit{getting started} section containing useful hints about how to configure.the application.

Further informations can also be found \href{https://www.fullstackpython.com/prometheus.html}{here}.

\textbf{2. Grafana}

The advantage of using \textbf{Grafana} for data visualization is to have a \textit{ready to use} interface to \textbf{Prometheus} available. Other possible solutions are not in focus here.

To install \textbf{Grafana} please visit the \href{https://grafana.com/}{homepage}.

\textbf{3. Prometheus Python client library}

This library belongs to the installation dependencies of \pkg. A separate installation is not required. In case you want to learn more about this client library please visit the following web pages:
\href{https://pypi.org/project/prometheus-client/}{[1]}, \href{https://prometheus.github.io/client_python/}{[2]}, \href{https://prometheus.io/docs/prometheus/latest/getting_started/}{[3]}.

% --------------------------------------------------------------------------------------------------------------

\newpage

\section{Configuration}

t.b.c.







